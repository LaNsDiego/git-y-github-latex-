% Generated by GrindEQ Word-to-LaTeX 
\documentclass{article} %%% use \documentstyle for old LaTeX compilers

\usepackage[spanish]{babel} %%% 'french', 'german', 'spanish', 'danish', etc.
\usepackage{amssymb}
\usepackage{amsmath}
\usepackage{txfonts}
\usepackage{mathdots}
\usepackage[classicReIm]{kpfonts}
\usepackage[dvips]{graphicx} %%% use 'pdftex' instead of 'dvips' for PDF output

% You can include more LaTeX packages here 


\begin{document}

%\selectlanguage{english} %%% remove comment delimiter ('%') and select language if required


\noindent 

\noindent 

\noindent 

\noindent 

\noindent INFORME DE LABORATORIO N$\mathrm{{}^\circ}$ XX - XYZ\textbf{ }\includegraphics*[width=0.47in, height=0.63in, keepaspectratio=false]{image1}

 
\[8\] 


\noindent \textbf{R2 fde}

\noindent \textbf{\includegraphics*[width=1.08in, height=1.46in, keepaspectratio=false]{image2}}

\noindent \textbf{}

\noindent \textbf{UNIVERSIDAD PRIVADA DE TACNA}

\noindent \textbf{}

\noindent \textbf{FACULTAD DE INGENIERIA}

\noindent \textbf{\textit{}}

\noindent \textbf{Escuela Profesional de Ingenier\'{i}a de Sistemas}

\noindent \textbf{}

\noindent 

\noindent \textbf{ INFORME DE LABORATORIO N� 01}

\noindent \textbf{``GIT Y GITHUB''}

\noindent \textbf{}

\noindent \textbf{ }

\noindent Curso: Base de Datos 2

\noindent \textbf{}

\noindent \textbf{}

\noindent Docente: Ing. Patrick Cuadros

\noindent \textbf{}

\noindent \textbf{}

\noindent \textbf{Layme Valeriano ,Diego Rolando(2017057865)}

\noindent \textbf{\textit{(Apellidos y nombres del estudiante (C\'{o}digo) - Arial N$\boldsymbol{{}^\circ}$ 18, en may\'{u}sculas y negrita)}}

\noindent 

\noindent 

\noindent 

\noindent 

\noindent 

\noindent \textbf{Tacna -- Per\'{u}}
\[2018\] 
\textbf{}

\noindent \textbf{}

\noindent \textbf{INDICE}

\noindent 

\noindent I. INFORMACI\'{O}N GENERAL 3- Objetivos: 3II. MARCO TEORICO 3III. PROCEDIMIENTO 3IV. ANALISIS E INTERPRETACION DE RESULTADOS (del evento pr\'{a}ctico) 4V. CUESTIONARIO (Opcional) 4

\noindent \textbf{\eject }

\noindent \textbf{\underbar{INFORME DE LABORATORIO N�  01  }}

\noindent \textbf{\underbar{TEMA:  GIT Y GITHUB}}

\noindent \underbar{}

\noindent 

\begin{enumerate}
\item  \textbf{INFORMACI\'{O}N GENERAL}

\item \textbf{ Objetivos:}
\end{enumerate}

\noindent Hacer una introducci\'{o}n y explicar que es GIT y GITHUB.

\begin{enumerate}
\item  \textbf{MARCO TEORICO}
\end{enumerate}

\noindent \textbf{Git}

\noindent Git, es un software de control de versiones dise\~{n}ado por Linus Torvalds.

\noindent 

\noindent \textbf{�Qu\'{e} es un control de versiones y por qu\'{e} deber\'{i}a utilizarlo?}

 rastrear el desarrollo y los cambios de tus documentos

 registrar los cambios que has hecho de una manera que puedas entender posteriormente

 experimentar con versiones distintas de un documento al mismo tiempo que conservas la m\'{a}s antigua

 fusionar dos versiones de un documento y administrar los conflictos existentes entre distintas versiones

 revertir cambios y volver atr\'{a}s gracias al historial de versiones anteriores de tu documento

\begin{enumerate}
\item  Ramificaci\'{o}n: nos permite realizar cambios en la estructura principal, pudiendo crear diferentes ramas sobre las que aplicar nuestras modificaciones en entornos aislados de la l\'{i}nea principal de desarrollo.
\end{enumerate}

\noindent 

\noindent \textbf{Los repositorios Git}

\noindent Esta clase de repositorios son una copia local del c\'{o}digo generado con una caracter\'{i}stica muy importante, y es que podemos hacer varias versiones para poder recular si nos hemos equivocado y nuestra aplicaci\'{o}n ya no funciona, o para trabajar en funcionalidades nuevas sin necesidad de modificar la versi\'{o}n funcional y as\'{i} no romper el proyecto.

\noindent 

\noindent 

\noindent \textbf{GITHUB}

\noindent GitHub es una forja (plataforma de desarrollo colaborativo) para alojar proyectos utilizando el sistema de control de versiones Git. Se utiliza principalmente para la creaci\'{o}n de c\'{o}digo fuente de programas de computadora.

\begin{enumerate}
\item  \textbf{PROCEDIMIENTO }
\end{enumerate}

\noindent \textbf{Ordenes b\'{a}sicas}

\noindent Iniciar un repositorio vac\'{i}o en unas carpeta espec\'{i}fica.

\noindent git init

\noindent A\~{n}adir un archivo especifico.

\noindent git add ``nombre\_de\_archivo''

\noindent A\~{n}adir todos los archivos del directorio

\noindent git add .

\noindent Confirmar los cambios realizados. El ``mensaje'' generalmente se usa para asociar al commit una breve descripci\'{o}n de los cambios realizados.

\noindent git commit --am ``mensaje''

\noindent Revertir el commit identificado por "hash\_commit"

\noindent git revert ``hash\_commit"

\noindent Subir la rama(branch) ``nombre\_rama'' al servidor remoto.

\noindent git push origin ``nombre rama''

\noindent Mostrar el estado actual de la rama(branch), como los cambios que hay sin hacer commit.

\noindent git status

\noindent 

\noindent \textbf{}

\noindent \textbf{}


\end{document}

